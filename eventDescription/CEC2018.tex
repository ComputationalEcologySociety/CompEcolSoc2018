\documentclass[10pt,english,]{article}
\usepackage[]{fourier}
\usepackage{setspace}
\setstretch{1}
\usepackage{amssymb,amsmath}
\usepackage{ifxetex,ifluatex}
\usepackage{fixltx2e} % provides \textsubscript
\ifnum 0\ifxetex 1\fi\ifluatex 1\fi=0 % if pdftex
  \usepackage[T1]{fontenc}
  \usepackage[utf8]{inputenc}
\else % if luatex or xelatex
  \ifxetex
    \usepackage{mathspec}
  \else
    \usepackage{fontspec}
  \fi
  \defaultfontfeatures{Ligatures=TeX,Scale=MatchLowercase}
\fi
% use upquote if available, for straight quotes in verbatim environments
\IfFileExists{upquote.sty}{\usepackage{upquote}}{}
% use microtype if available
\IfFileExists{microtype.sty}{%
\usepackage{microtype}
\UseMicrotypeSet[protrusion]{basicmath} % disable protrusion for tt fonts
}{}
\usepackage[margin=1in]{geometry}
\usepackage{hyperref}
\hypersetup{unicode=true,
            pdftitle={Computationnal Ecology Conference},
            pdfborder={0 0 0},
            breaklinks=true}
\urlstyle{same}  % don't use monospace font for urls
\ifnum 0\ifxetex 1\fi\ifluatex 1\fi=0 % if pdftex
  \usepackage[shorthands=off,main=english]{babel}
\else
  \usepackage{polyglossia}
  \setmainlanguage[]{english}
\fi
\usepackage{graphicx,grffile}
\makeatletter
\def\maxwidth{\ifdim\Gin@nat@width>\linewidth\linewidth\else\Gin@nat@width\fi}
\def\maxheight{\ifdim\Gin@nat@height>\textheight\textheight\else\Gin@nat@height\fi}
\makeatother
% Scale images if necessary, so that they will not overflow the page
% margins by default, and it is still possible to overwrite the defaults
% using explicit options in \includegraphics[width, height, ...]{}
\setkeys{Gin}{width=\maxwidth,height=\maxheight,keepaspectratio}
\IfFileExists{parskip.sty}{%
\usepackage{parskip}
}{% else
\setlength{\parindent}{0pt}
\setlength{\parskip}{6pt plus 2pt minus 1pt}
}
\setlength{\emergencystretch}{3em}  % prevent overfull lines
\providecommand{\tightlist}{%
  \setlength{\itemsep}{0pt}\setlength{\parskip}{0pt}}
\setcounter{secnumdepth}{5}
% Redefines (sub)paragraphs to behave more like sections
\ifx\paragraph\undefined\else
\let\oldparagraph\paragraph
\renewcommand{\paragraph}[1]{\oldparagraph{#1}\mbox{}}
\fi
\ifx\subparagraph\undefined\else
\let\oldsubparagraph\subparagraph
\renewcommand{\subparagraph}[1]{\oldsubparagraph{#1}\mbox{}}
\fi

%%% Use protect on footnotes to avoid problems with footnotes in titles
\let\rmarkdownfootnote\footnote%
\def\footnote{\protect\rmarkdownfootnote}

%%% Change title format to be more compact
\usepackage{titling}

% Create subtitle command for use in maketitle
\newcommand{\subtitle}[1]{
  \posttitle{
    \begin{center}\large#1\end{center}
    }
}

\setlength{\droptitle}{-2em}
  \title{Computationnal Ecology Conference}
  \pretitle{\vspace{\droptitle}\centering\huge}
  \posttitle{\par}
  \author{}
  \preauthor{}\postauthor{}
  \predate{\centering\large\emph}
  \postdate{\par}
  \date{01-10-2017}


\begin{document}
\maketitle

\section{Computational Ecology
Conference}\label{computational-ecology-conference}

Computational ecology is now recognized as an
\href{http://rsfs.royalsocietypublishing.org/content/2/2/241}{emerging
science}, and the time is ripe for a conference gathering young and
innovative minds currently dedicating their research acumen to the
development of this exciting field. Organized by and tailored for early
career scientists, the event will offer an experience allowing
participants to share research and ideas in a variety of ways through a
mix between educational workshops, expert panel discussions, conferences
and interactive coding sessions.

Over a full week of activities in the beautiful city of Montreal,
participants will first be invited to participate to educational
workshops. Beginners will have the opportunity to participate to a full
two day \href{https://software-carpentry.org/}{Software Carpenty}
workshop, while more advanced participants will be able to choose from a
selection of \href{http://www.datacarpentry.org/}{Data Carpentry}
workshops and special topics in advanced R programming.

The third day will be dedicated to discovering the current state and the
future of computational ecology as an emerging science through
participant conferences (\emph{i.e.} posters and 5 minutes
presentations) and expert panel discussions.

Finally, the last two days of the event will be held in collaboration
with
\href{https://www.meetup.com/DataforGood-Montreal/?_cookie-check=FowCtx-gNWyZzv7X}{Data
for Good} in the form of a two day ``hackathon''. Projects will aim at
tackling computational ecology data problems alongside data scientists
from the Montreal region and applied to non profit organizations.
Participants will also be invited to participate to a special issue in
that will be published following this event.

Number of participants will be limited and participants will be selected
through a competitive selection process.


\end{document}
